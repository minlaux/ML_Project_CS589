\documentclass{article}

\usepackage[utf8]{inputenc}
\usepackage{geometry}
\usepackage{graphicx}
\usepackage{amsmath}
\usepackage{amsfonts}
\usepackage{hyperref}
\usepackage{caption}

\usepackage{booktabs}
\usepackage[table]{xcolor}
\usepackage{graphicx}
\usepackage{xcolor}
% \usepackage[table,xcdraw]{xcolor}
\usepackage{colortbl}
%removes indent
\setlength{\parindent}{0pt}

% Document Setup
\title{Exploring Machine Learning Model Performance across Diverse Datasets: A Comparative Analysis}
\author{Yan Mazheika \and Polina Petrova}
\date{\today}

\begin{document}
\maketitle

\section*{Introduction}
This report provides an analysis of using various machine learning algorithms on various datasets.
The goal is to identify the best-use scenarios of model variants given the nature of input data.
In this report, the authors analyze the
\begin{itemize}
    \item Decision Tree, \textit{provided by Polina Petrova}
    \item k Nearest Neighbors, \textit{provided by Polina Petrova}
    \item Neural Network, \textit{provided by Yan Mazheika}
    \item Random Forrest, \textit{provided by Yan Mazheika}
\end{itemize}
against the handwritten digits, titanic survival, loan eligibility, and Parkinson's classification datasets.
\\

It is our aim to explain the nature of the data, our models, and the performance of these classifiers. Throughout our report,
we justify our algorithm choice for the given dataset and our choice of hyper-parameters for the tuning of the algorithm. These insights
are for the reader's benefit; we also aim to provide insight into how to solve novel machine problems, which algorithms may work best, and how
to adjust their hyper-parameters for optimal performance.

\subsection*{Approach}
The authors have chosen to coordinate on every dataset. For the first two algorithms of every dataset, one of them comes from \textit{Yan Mazheika}
while the other comes from \textit{Polina Petrova}. These algorithms may have been modified to handle a new type of data
but their fundamental logic stays consistent from previous use cases.
\\

We use cross-validation of k=10 folds when testing the performance of our models. We also transform the data into the [0,1] range when using the neural network or kNN classifiers.


\newgeometry{left=1cm,right=1cm}

\vspace*{\fill}

\begin{center}
    {\large Model Performance on Diverse Datasets}
    \begin{table}[h]
    \begin{tabular}{l|cc|cc|cc|cc}
    Dataset:             & \multicolumn{2}{c|}{Digits}                                   & \multicolumn{2}{c|}{Titanic}                                  & \multicolumn{2}{c|}{Loan}                                     & \multicolumn{2}{c}{Parkinson's}                               \\ \hline
                         & Accuracy                      & F1-Score                      & Accuracy                      & F1-Score                      & Accuracy                      & F1-Score                      & Accuracy                      & F1-Score                      \\ \cline{2-9} 
    k-NN                 & .8698                         & .9298                         & -                             & -                             & -                             & -                             & .7526                        & .8557                             \\
    Decision Tree        & -                             & -                             & .7023                         & .8234                         & .5354                         & \cellcolor[HTML]{C0C0C0}.6860 & -                             & -                             \\
    Random Forrest       & -                             & -                             & -                             & -                             & \cellcolor[HTML]{C0C0C0}.7354 & .6650                         & -                             & -                             \\
    Neural Network       & \cellcolor[HTML]{C0C0C0}.9962 & \cellcolor[HTML]{C0C0C0}.9809 & \cellcolor[HTML]{C0C0C0}.9663 & \cellcolor[HTML]{C0C0C0}.9652 & -                             & -                             & \cellcolor[HTML]{C0C0C0}.9226 & \cellcolor[HTML]{C0C0C0}.8940 \\
    Network Forrest & .8255                            & .5710                             & .5808                             & .6968                             & .6917                             & .4084                             & .7542                             & .4281                            
    \end{tabular}
    \end{table}
\end{center}

\vfill

\restoregeometry


\subsection*{An Additional Variant}
Consider a model variant which essentially performs logistic regression on each feature and then performs a vote on the output of each feature.
The neural net architecture was modified to have a single neuron in one hidden layer to imitate logistic regression. We perform this logistic regression for each feature
and then perform a vote. Afterwards, a k-NN variant gets to vote, which potentially breaks ties.
\\

When each of the neural networks makes a vote, we take the vote or votes with the highest frequency. These represent the features which contributed to the majority vote.
If the vote is evenly split, all of the features are used in the k-NN algorithm to break the tie. Otherwise, it's just the features that contributed to the majority vote.
This k-NN variant performs the euclidean distance calculation to find the closest 25\% of the training data to contribute to the vote.
\\

The idea here is to rank the features by their importance in the dataset. If they contributed to the majority vote, we essentially confirm
that they are important. In a way, it's like a decision tree in the way we prioritize the most important features. However, it's important
to note that unlike a decision tree, we don't have access to the test dataset to confirm that they actually are. So essentially, we are estimating the 
best features to 'split' on rather than tuning them with the y-values.
\\

We'll analyze this new variant over all of the datasets we look at. This graphic displays an example:
\begin{center}
    \includegraphics*[width=0.8\textwidth]{./src/figures/variant.png}
\end{center}

\textcolor{red}{This is our attempt for extra credit \#3 and \#4. Because we analyze this variant over all of the datasets, this is also for extra credit \#1.}


\newpage
\section*{Digits Dataset}

The digits dataset comes from the sci-kit learn library, available in Python. We have chosen to analyze the performance of a neural network classifier and the k-Nearest Neighbors classifier. Again, we mention
that the neural network tuning was performed by \textit{Yan Mazheika} while the kNN tuning was done by \textit{Polina Petrova}.
\\

We chose a neural network as one of our algorithms for this dataset because of the algorithm's ability to
handle complex patterns and identify non-linear patterns in these data. Anecdotally, neural networks have shown
impressive results in image classification tasks, making them a top contender for handwritten digit recognition, the focus of this dataset.
\\

On the other hand, the k-NN algorithm is a simple yet effective classifier that makes predictions based on the closeness of past training instances in the feature space. We figured that similar pixel activations in the handwritten digits
would translate into less distance in the 64-dimensional feature space. A concern we had when choosing this algorithm is precision loss; kNN performs poorly with a large number of features, which is 64 in our case. The distance measurement between two instances
converges to zero as we add more features and normalize them. However, this wasn't a major problem in our analysis.

\subsection*{Tuning Hyper-parameters}
For the neural network, we've chosen an $\alpha$ of 0.05 and a regularization constant of $\lambda=0.01$. At first, with a $\lambda=0$ and $\alpha=0.50$, the model had good performance but a quick convergence in the training curve.
We found increased performance and a slower convergence by decreasing $\alpha$ and increasing $\lambda$ to a final accuracy of 99.62\%. This indicates a misclassification of 6 total instances out of 1,797 total.
The architecture was chosen because it had the lowest accuracy variance between each fold out of the models tested.
\\

In our implementation of the k-NN algorithm, we tested different (integer) values of $k$ in the range [1, 52) for an ideal hyperparameter. 
We observed an initial jump in the evaluation metrics between $k$ = 1 and $k$ = 2 and a high variance.
This points to the algorithm being sensitive to outliers at very small values of $k$.
As the value of $k$ increases, the values of the evaluation metrics increase, as well, and stabilise by $k$ = 51 at 86.98\% for accuracy and 92.98\% for F1-score.
Increasing this hyperparameter value also led to smaller variance, which is helpful when dealing with noisy datasets.
In witnessing this convergence, we have chosen a final hyperparameter value for the k-NN algorithm to be $k$ = 51.
With a larger $k$ value and smoother decision boundary, we mitigate the chance of overfitting our model to the data.
\\

The network forrest was tuned for runtime rather than for accuracy. We found that the model performed well with a learning rate of 0.1 and a regularization constant of 0. Any regularization above 0 would result in a model that performed poorly. This is likely due to the fact that the model is not complex enough to require regularization.
It's worth noting that the model takes significantly longer to train than the other models, especially the neural network because this model does not take advantage of parallel processing or vectorization.


\newgeometry{left=1cm,right=1cm}
\subsection*{Performance Metrics}
%we can adjust for three models later
\begin{center}
    \texttt{Neural Network}

    \includegraphics*[width=0.6\textwidth]{./src/figures/Digits-test-cost.png}
\end{center}

\begin{minipage}{0.49\textwidth}
    \centering

    \begin{tabular}{lc}
        \toprule
        \multicolumn{2}{c}{Neural Network} \\
        \midrule
        Learning Rate $\alpha$ & 0.0500 \\
        Regularization $\lambda$ & 0.0100 \\
        Architecture & [64, 32, 32, 10] \\
        Mean Accuracy & 0.9962 \\
        Mean F1-score & 0.9809 \\
        Mean Test Cost & 59.3100 \\
        \bottomrule
    \end{tabular}
\end{minipage}
\hfill
\begin{minipage}{0.49\textwidth}
    \centering
    
    \includegraphics*[width=1\textwidth]{./src/figures/Digits_train_cost.png}
    \captionof*{figure}{\textit{Training cost over 179,700 instances}}
    \vfill
\end{minipage}

\newpage
\begin{center}
    \texttt{k-NN Algorithm}
\end{center}
\begin{minipage}{0.49\textwidth}
    \centering

    \includegraphics*[width=\textwidth]{./src/figures/Accuracy Digits.png}

    \includegraphics*[width=\textwidth]{./src/figures/F-Score Digits.png}
    
    \begin{tabular}{lc}
        \toprule
        \multicolumn{2}{c}{k-NN} \\
        \midrule
        Neighbors $k$ & 51 \\
        Mean Accuracy & 0.8698 \\
        Mean F1-score & 0.9298 \\
        \bottomrule
    \end{tabular}
    \captionof*{figure}{\textit{Metrics for Preferred Model Variant}}
\end{minipage}
\begin{minipage}{0.49\textwidth}
    \centering
    \vfill
    \begin{tabular}{lrrr}
        \toprule
        $k$ & Test Accuracy & Test F-Score \\
        \midrule
        1 & {0.7291} & {0.8413} \\
        3 & {0.7615} & {0.8625} \\
        5 & {0.7615} & {0.8629} \\
        7 & {0.7799} & {0.8748} \\
        9 & {0.7872} & {0.8797} \\
        11 & {0.7911} & {0.8821} \\
        13 & {0.7939} & {0.8836} \\
        15 & {0.8000} & {0.8875} \\
        17 & {0.8073} & {0.8921} \\
        19 & {0.8089} & {0.8932} \\
        21 & {0.8184} & {0.8991} \\
        23 & {0.8235} & {0.9022} \\
        25 & {0.8307} & {0.9066} \\
        27 & {0.8302} & {0.9062} \\
        29 & {0.8330} & {0.9079} \\
        31 & {0.8380} & {0.9111} \\
        33 & {0.8397} & {0.9121} \\
        35 & {0.8441} & {0.9148} \\
        37 & {0.8503} & {0.9184} \\
        39 & {0.8542} & {0.9208} \\
        41 & {0.8547} & {0.9209} \\
        43 & {0.8570} & {0.9223} \\
        45 & {0.8615} & {0.9249} \\
        47 & {0.8670} & {0.9281} \\
        49 & {0.8682} & {0.9287} \\
        51 & {0.8698} & {0.9298} \\
        \bottomrule
    \end{tabular}
    \captionof*{figure}{\textit{Performance Metrics over various k}}
    \vfill
\end{minipage}
\restoregeometry
\begin{center}
    \texttt{Network Forrest}
\end{center}
\begin{minipage}{0.49\textwidth}
    \centering
    \includegraphics*[width=\textwidth]{./src/figures/Digits dataset_variant.png}
\end{minipage}
\hfill
\begin{minipage}{0.49\textwidth}
    \centering
    \begin{tabular}{lr}
        \toprule
         & Model 1 \\
        \midrule
        Learning Rate $\alpha$ & 0.1000 \\
        Regularization $\lambda$ & 0.0000 \\
        Mean Accuracy & 0.8255 \\
        Mean F1-score & 0.5710 \\
        Number of Iterations & 20.0000 \\
        Batch Size & 10.0000 \\
        \bottomrule
        \end{tabular}
\end{minipage}

\subsection*{Analysis}
Reporting on the final runs of the neural network and k-NN algorithms on the digits dataset, both models performed well on the testing instances.
However, testing on the neural network resulted in higher accuracy and F1-score values as opposed to when testing on k-NN.
Dealing with an 8x8 pixel image, k-NN is able to effectively measure distances in the feature space when making predictions.
Although k-NN performs well on this dataset, it struggles in distinguishing between two or more digits that look alike when handwritten.
For example, an analysis of its predictions made on an instance of class 0 may return a substantial number of predictions of class 9, which would be a factor in decreasing accuracy.
\\

The neural network acts as a well-defined alternative to this issue, as it captures more complex patterns and data variability present in handwriting.
Additionally, a higher neuron count facilitates learning these handwriting variations of numbers of the same class, making the network more reliable in predicting instances that look alike and presenting a higher accuracy.
\\

Additionally the network forrest, our variant model, performed surprisingly well on this dataset. This is likely due to the presence of features that
are uniformly important. Each individual pixel does not decide the photo, but the majority of them do. This is why the model performed well in this context.
It's worth noting that with 64 features and a relatively high accuracy, it is almost certain that the k-NN variant tie-breaker was not used.
Also, what benefits our model is the low resolution of the images, allowing for digits of one class to not vary much in the 64-pixel feature space.

\newpage
\section*{Titanic Dataset}
We pre-processed some of the titanic dataset. At first, there was a 'Sex' column which was encoded into a 'Male' column indicating a 1 if the 'Sex' of a given instance contained 'male' and a 0 if an instance contained 'female'.
There was also a 'Name' column which was removed entirely. At first, we attempted to process this by encoding every English letter into a number. This resulted in a very large data-frame that led to poor results after processing. 
\\

Afterwards, we tried to one-hot encode the first word of the 'Name' column which indicated the title of a person. Because this dataset represents a time when a person's title had greater significance, we figured that there would be differences in the survival rates between prestigious and non-prestigious titles (such as 'Rev' or 'Master' versus 'Mr' or 'Miss').
We also hypothesized that persons with the title 'Rev,' which indicates religious affiliation, would be less likely to survive as we thought that their role as a religious figure would influence them to stay on the ship for longer than others. Indeed, every reverend in the dataset did not survive.
\\

Empirically, we found that this didn't work well, which led to the removal of the 'Name' feature in the dataset.
Perhaps this is because 'Pclass' contains the ticket class of a given passenger, which mitigates the role that the 'Name' column plays in indicating the socio-economic status of a passenger. Additionally,
the added dimensionality of encoding around 8 titles could have mitigated the importance of the un-altered columns.
\\

The choice of applying a neural network for this task can be justified by its performance (speed) and ability to learn complex patterns. 
It is arguably the easiest model to tune with respect to the datasets this paper presents.
\\

The choice of applying a decision tree for this task is natural, as well. There are a number of more important features that a high-bias model can benefit from. These
include passenger class, ticket fare, and gender. With three main features impacting survivability, we can benefit from a decision tree by quickly sub-setting the dataset on those features.

\subsection*{Tuning Hyper-parameters}

The neural network was analyzed over various learning rates and regularization constants. We found that a learning rate of 0.04 and a regularization constant of 0.09 produced the best results.
We found that a larger-than-usual regularization constant was necessary to reducing the variance in the model's predictions. At lower $\lambda$ values, the model appeared to overfit the data, which was evident by the high variance in the training cost curve.
A smaller architecture was chosen to reduce the complexity of the model as performance suffered with the presence of more than one hidden layer.
This is likely due to the complexity of the dataset, which features rather arbitrary social factors.
\\

Conducting multiple analyses of the decision tree algorithm on the titanic dataset with differing maximum tree depths,
we have determined the preferred hyperparameter to be $max\_depth$ = 4. 
At this value, accuracy and F1-score reach a peak at 70.23\% and 82.34\%, respectively, after which both metrics decrease substantially.
Deeper trees lead to overfitting and decrease in generalisation ability, 
which is evidenced by the sharp increace in variance on both accuracy and F-1 score after tree depth 4.


\subsection*{Performance Metrics}
\begin{center}
    \texttt{Neural Network}

    \includegraphics*[width=0.8\textwidth]{./src/figures/Titanic-test-cost.png}
\end{center}

\begin{minipage}{0.49\textwidth}
    \centering

    \begin{tabular}{ll}
        \toprule
        \multicolumn{2}{c}{Neural Network} \\
        \midrule
        Learning Rate $\alpha$ & 0.0400 \\
        Regularization $\lambda$ & 0.0900 \\
        Architecture & [6, 4, 2] \\
        Mean Accuracy & 0.9663 \\
        Mean F1-score & 0.9652 \\
        Mean Test Cost & 84.8500 \\
        \bottomrule
    \end{tabular}
    
\end{minipage}
\hfill
\begin{minipage}{0.49\textwidth}
    \centering
    \includegraphics*[width=0.9\textwidth]{./src/figures/Titanic_train_cost.png}
    \captionof*{figure}{\textit{Training cost over 87,700 instances}}
\end{minipage}

\newgeometry{left=1cm,right=1cm}
\begin{center}
    \texttt{Decision Tree}
\end{center}

\begin{minipage}{0.49\textwidth}
    \centering 

    \includegraphics*[width=\textwidth]{./src/figures/Accuracy Titanic.png}
    \includegraphics*[width=\textwidth]{./src/figures/F-Score Titanic.png}

\end{minipage}
\begin{minipage}{0.49\textwidth}
    \centering
    \begin{tabular}{rrr}
        \toprule
        Maximum depth & Test Accuracy & Test F-score \\
        \midrule
        1 & {0.7852} & {0.8793} \\
        2 & {0.6386} & {0.7774} \\
        3 & {0.5898} & {0.7372} \\
        4 & {0.7125} & {0.8291} \\
        5 & {0.6784} & {0.7920} \\
        6 & {0.6295} & {0.7540} \\
        7 & {0.5227} & {0.6619} \\
        8 & {0.4216} & {0.5802} \\
        9 & {0.4216} & {0.5802} \\
        \bottomrule
    \end{tabular}


    \vspace*{30pt}
    \begin{tabular}{lc}
        \toprule
        \multicolumn{2}{c}{Decision Tree} \\
        \midrule
        Maximum Depth & 4 \\
        Mean Accuracy & 0.7023 \\
        Mean F1-score & 0.8234 \\
        \bottomrule
    \end{tabular}
    \captionof*{figure}{\textit{Metrics for Preferred Model Variant}}
\end{minipage}
\restoregeometry
\begin{center}
    \texttt{Network Forrest}
\end{center}
\begin{minipage}{0.49\textwidth}
    \centering
    \includegraphics*[width=\textwidth]{./src/figures/Titanic Dataset_variant.png}
\end{minipage}
\hfill
\begin{minipage}{0.49\textwidth}
    \centering
    \begin{tabular}{lr}
        \toprule
         & Model 1 \\
        \midrule
        Learning Rate $\alpha$ & 0.0500 \\
        Regularization $\lambda$ & 0.0000 \\
        Mean Accuracy & 0.5808 \\
        Mean F1-score & 0.6968 \\
        Number of Iterations & 100.0000 \\
        Batch Size & 10.0000 \\
        \bottomrule
        \end{tabular}
\end{minipage}

\subsection*{Analysis}
Our decision tree struggled in providing reliable predictions on the titanic dataset. 
This is possibly due to the fact that the dataset contained numerical attributes such as $Ticket\ Fare$ and $Age$
that may be difficult to evaluate on this model.
The decision nodes were split based on a numerical threshold that maximised information gain at each decision point.
This approach is particularly challenging given the large range of numerical values present in the data.
For example, there were passengers as old as 80 years-old on the ship, along with infants as young as 5 months-old.
Similarly, the ticket fare values ranged from 0.0 to 512, showing a wide distribution of values. 
Although we employed feature normalisation, this variability in numerical attributes could have contributed to 
the difficulty in finding optimal splits during the tree construction process. 
Additionally, deeper trees tend to create more decision nodes based on numerical thresholds, 
which may struggle to effectively differentiate between instances with similar normalised values but different original scales.
\\

On the other hand, employing a neural network on this dataset proved to be substantially more accurate in predicting instance classes.
However, we observed a concerning trend in the neural network's performance metrics. 
While the accuracy on the test set was high, we witnessed a very high mean test cost, as well.
Furthermore, the extreme variance in cost as the model processed more instances
indicates that the neural network's predictions were inconsistent across different subsets of the dataset.
This may be a consequence of class imbalance in the titanic dataset.
The proportion of the class $Did\ Not\ Survive$ was 61\%, 
while for the class $Survived$, the proportion was only 38\%.
Our neural network model may be prioritising minimising errors on the majority class, 
leading to higher costs for misclassifying the minority class instances.
\\

Unsurprisingly, the network forrest model performed poorly on this dataset. This is because of the aforementioned non-uniform importance of each feature.
The model performed poorly on this dataset because the features were not uniformly important. This is again due to the fact that there are key features that are more important than others.
Our model is not adequately able to learn any new information, performing poorer than a random classifier.


\newpage
\section*{Loan Eligibility Dataset}
The loan eligibility dataset is a binary classification problem. We have chosen to analyze the performance of a decision tree and a random forest classifier.
\\

The choice of applying a decision tree to the problem was natural. Anecdotally, decision trees are usually taught in the context of loan eligibility problems. 
This is because decision trees are based on a series of if-else statements that are both easy to interpret and prioritize the most important features.
In the context of this dataset, the presence of 11 total features makes some features more important than others, which will ideally be filtered for relevance by a decision tree.
\\

The choice of applying a random forrest was also sensible. Random forests are ensembles of decision trees and aim to reduce the variance of output problems.
With this algorithm, we can ideally expect a model less prone to over-fitting compared to a decision tree if the nature of the data is nuanced.
\\

We had to pre-process this dataset as well. The provided file contained a Loan\_ID container which the models would misinterpet as a feature and end up memorizing the target class as
as a function of the Loan\_ID. Therefore, we removed this column from the dataset. To successfully process the dataset through the random forrest algorithm, the classes had to be encoded as numbers
but none of the features were one-hot encoded. This is because the random forrest algorithm can handle categorical data without the need for this. For instance, the graduate feature was encoded as 0 for 'No' and 1 for 'Yes'.

\subsection*{Tuning Hyper-parameters}
Similar to the implementation of the decision tree model for the titanic dataset,
the performance of the model diminished with maximum depth values above 2.
Thus, we chose the hyperparameter $max\_depth$ = 2 to represent this model.
Furthermore, we found that repeated testing with maximum depth values of 2 and 3
produced similar evaluation values.
The decision to select $max\_depth$ = 2 stemmed from the oscillation in performance
metrics during repeated testing on $max\_depth$ = 3, 
which produced greater instability than the former hyperparameter value. 
\\

We found that the random forest model performed best with 2 ensembles trees. Perhaps other values were optimal such as $ntree = 20$, as seen by the graph below, but
the variant with the highest accuracy was preferred despite the peak F1-score being at $ntree=20$ and a relatively high accuracy, as well.
The stopping condition of each tree in the random forrest was the same: a minimal split entropy of 0.21. This was chosen through the method of moments, simply the value that produced the best results at 5 decision trees.
Alternative stopping conditions that were tested with cross validation include the minimal split entropy, ranging from 0.01 to 0.27 in 10 equal step sizes, the maximum depth of each decision tree, ranging from 3 to 22 in 10 equal steps,
and the proportion of the dataset required in each branch to perform a split, which ranged from 1\% to 27\% in 10 equal steps.
\\

Tuning the network forrest proved an impossible task. The model was unable to perform better than a random classifier over various different hyperparameters.

\newgeometry{left=1cm,right=1cm}
\subsection*{Performance Metrics}
\begin{center}
    \texttt{Decision Tree}
\end{center}

\begin{minipage}{0.49\textwidth}
    \centering 
    
    \includegraphics*[width=\textwidth]{./src/figures/Accuracy Loan.png}
    \includegraphics*[width=\textwidth]{./src/figures/F-Score Loan.png}

    \centering
    \begin{tabular}{lc}
        \toprule
        \multicolumn{2}{c}{Decision Tree} \\
        \midrule
        Maximum Depth & 2 \\
        Mean Accuracy & 0.5354 \\
        Mean F1-score & 0.6860 \\
        \bottomrule
    \end{tabular}
    \captionof*{figure}{\textit{Metrics for Preferred Model Variant}}
\end{minipage}
\begin{minipage}{0.49\textwidth}
    \centering
    \begin{tabular}{rrr}
        \toprule
        Maximum depth & Test Accuracy & Test F-score \\
        \midrule
        1 & {0.8083} & {0.8930} \\
        2 & {0.6458} & {0.7657} \\
        3 & {0.4708} & {0.6357} \\
        4 & {0.3583} & {0.5139} \\
        5 & {0.3104} & {0.4606} \\
        6 & {0.3042} & {0.4609} \\
        11 & {0.2437} & {0.3894} \\
        16 & {0.2437} & {0.3894} \\
        21 & {0.2437} & {0.3894} \\
        26 & {0.2437} & {0.3894} \\
        31 & {0.2437} & {0.3894} \\
        36 & {0.2437} & {0.3894} \\
        41 & {0.2437} & {0.3894} \\
        46 & {0.2437} & {0.3894} \\
        51 & {0.2437} & {0.3894} \\
        56 & {0.2437} & {0.3894} \\
        61 & {0.2437} & {0.3894} \\
        66 & {0.2437} & {0.3894} \\
        71 & {0.2437} & {0.3894} \\
        76 & {0.2437} & {0.3894} \\
        81 & {0.2437} & {0.3894} \\
        86 & {0.2437} & {0.3894} \\
        91 & {0.2437} & {0.3894} \\
        96 & {0.2437} & {0.3894} \\
        \bottomrule
    \end{tabular}
\end{minipage}
\restoregeometry

\newgeometry{left=1cm,right=1cm}
\begin{center}
    \texttt{Random Forrest}
\end{center}
\begin{minipage}{0.49\textwidth}
    \centering

    \includegraphics*[width=\textwidth]{./src/figures/Loan-accuracy_final.png}
    
\end{minipage}
\hfill
\begin{minipage}{0.49\textwidth}
    \centering
    \includegraphics*[width=\textwidth]{./src/figures/Loan-f1-score_final.png}
\end{minipage}

\begin{center}
    \begin{tabular}{rrrrr}
    \toprule
    accuracy & precision & recall & f1-score & ntree \\
    \midrule
    0.6667 & 0.5729 & 0.5493 & 0.5609 & 1 \\
    0.7354 & 0.6875 & 0.6440 & 0.6650 & 2 \\
    0.7271 & 0.6755 & 0.6286 & 0.6512 & 5 \\
    0.7208 & 0.7436 & 0.5585 & 0.6379 & 10 \\
    0.7292 & 0.8348 & 0.5627 & 0.6722 & 20 \\
    0.7021 & 0.6756 & 0.5281 & 0.5928 & 30 \\
    0.7104 & 0.7538 & 0.5360 & 0.6265 & 50 \\
    \bottomrule
    \end{tabular}
    \captionof*{figure}{\textit{Performance metrics across various $n$-tree}}
\end{center}
\restoregeometry
\begin{center}
    \texttt{Network Forrest}
\end{center}
\begin{minipage}{0.49\textwidth}
    \centering
    \includegraphics*[width=\textwidth]{./src/figures/Loans Dataset_variant.png}
\end{minipage}
\hfill
\begin{minipage}{0.49\textwidth}
    \centering
    \begin{tabular}{lr}
        \toprule
         & Model 1 \\
        \midrule
        Learning Rate $\alpha$ & 0.0500 \\
        Regularization $\lambda$ & 0.0000 \\
        Mean Accuracy & 0.6917 \\
        Mean F1-score & 0.4084 \\
        Number of Iterations & 100.0000 \\
        Batch Size & 10.0000 \\
        \bottomrule
        \end{tabular}
\end{minipage}


\subsection*{Analysis}
For the loan dataset, we employed the decision tree and random forest algorithms.
It is crucial to note that the random forest model used entropy as a stopping criterion,
while the decision tree model used tree depth as its stopping criterion.
This may be a factor in why the random forest classifier performed better on both accuracy and F1-score
than the decision tree classifier.
That being said, both models converged to poor local optimums, 
indicating a lapse in classification ability for tree-based algorithms on this dataset.
\\

It would be difficult to compare the effect of adding an extra decision tree to result in the random forrest model because the algorithms are tuned to different stopping conditions.
However, the higher performance of the random forrest model suggests that an ensemble is beneficial, given the nature of the data. This suggests that the data is nuanced and requires greater complexity to capture 
less-subtle patterns. Indeed, the relatively high performance of the random forest at $ntree=20$ suggests that this is the case and that this alternative model variant could also be acceptable.
The relatively similar precision and recall scores for the 2-tree ensemble suggests that the model is classifying false positives and negatives at similar rates. This is interesting since 67\% of the dataset is of class 'Y'.
This potentially explains the precision and recall metrics of the model at $ntree=20$, where the model is classifying false negatives at a higher rate than false positives.
\\

Again, we see the network forrest struggling to perform. Here, this model acts as a statistical test against for the non-uniformity of the features. 
The majority vote for each feature does not perform well in this context.

\newpage
\section*{Oxford Parkinson's Disease Dataset}
The Parkinson's dataset proved to be very tricky. Again, we used a neural network and a k-Nearest Neighbors algorithm, as we did with the digits dataset.
Having a large number of features, the k-NN algorithm was expected to perform poorly and the neural network was expected to perform well.
Indeed, this was the case and unlike with the digits dataset, our hypothesis was confirmed.
\\

The nature of the data is biological and complex by nature. It seems that some features are simply more important than others are. 
This explains why both the network forrest and k-NN algorithm perform poorly.

\subsection*{Tuning Hyper-parameters}

The k-NN algorithm witnessed a large dip in performance between $k$ values of 1 and 56.
The high variance observed in predictions at these values may be indicative of 
the algorithm's sensitivity to local fluctuations in the training data. 
At lower values of $k$, the algorithm tends to overfit to the noise in the training data, 
resulting in poor generalisation performance on the test set,
which is what we witnessed with our metrics dipping below 26\% for accuracy and below 40\% for F1-score.
At $k$ = 61, the algorithm finally converged to an optimum at 75.26\% accuracy and 85.57\% F1-score.
As this was the first $k$ value to produce stable results, we selected it for our hyperparameter, 
looking to mitigate the overfitting phenomenon observed with smaller values of $k$.
\\

Tuning the neural network was also quite tricky. We found out that the model did not perform well with any significant regularization.
Additionally, the nature of the data was such that any changes in hyper-parameters would drastically change model performance.
The ideal learning rate was found to be 0.1, which was visibly checked to be valid by the training and testing cost curves.
Furthermore, the network performed poorly with any number of hidden layers greater than one. Better results were also found with a smaller number of nodes in
the single hidden layer, suggesting that certain features were more important than others.
\\

Tuning the network forrest was again optimized primarily for processing speed. An $\alpha$ of 0.02 was chosen because it produced the best results efficiently.
The sensitivity of the dataset suggests that even lower values of alpha over a larger number of iterations is preferable.

\newgeometry{left=1cm,right=1cm}
\subsection*{Performance Metrics}

\begin{center}
    \texttt{k-NN Algorithm}
\end{center}
\begin{minipage}{0.49\textwidth}
    \centering

    \includegraphics*[width=\textwidth]{./src/figures/Accuracy Parkinsons.png}
    \includegraphics*[width=\textwidth]{./src/figures/F-Score Parkinsons.png}

    \begin{tabular}{lc}
        \toprule
        \multicolumn{2}{c}{k-NN} \\
        \midrule
        Neighbors $k$ & 61 \\
        Mean Accuracy & 0.7526 \\
        Mean F1-score & 0.8557 \\
        \bottomrule
    \end{tabular}
    \captionof*{figure}{\textit{Metrics for Preferred Model Variant}}
\end{minipage}
\begin{minipage}{0.49\textwidth}
    \centering
    \begin{tabular}{rrr}
    \toprule
        k & Test Accuracy & Test F-Score \\
        \midrule
        1 & {0.5947} & {0.7367} \\
        6 & {0.2789} & {0.4319} \\
        11 & {0.2789} & {0.4319} \\
        16 & {0.2474} & {0.3921} \\
        21 & {0.2474} & {0.3893} \\
        26 & {0.2474} & {0.3893} \\
        31 & {0.2474} & {0.3893} \\
        36 & {0.2474} & {0.3893} \\
        41 & {0.2579} & {0.4030} \\
        46 & {0.4211} & {0.5625} \\
        51 & {0.6000} & {0.7320} \\
        56 & {0.7211} & {0.8350} \\
        61 & {0.7526} & {0.8561} \\
        66 & {0.7526} & {0.8561} \\
        71 & {0.7526} & {0.8561} \\
        76 & {0.7526} & {0.8561} \\
        81 & {0.7526} & {0.8561} \\
        86 & {0.7526} & {0.8561} \\
        91 & {0.7526} & {0.8561} \\
        96 & {0.7526} & {0.8561} \\
        101 & {0.7526} & {0.8561} \\
    \bottomrule
    \end{tabular}
\end{minipage}
\restoregeometry

\begin{center}
    \texttt{Neural Network}

    \includegraphics*[width=0.8\textwidth]{./src/figures/Parikinson's-test-cost.png}
\end{center}

\begin{minipage}{0.49\textwidth}
    \centering
    
    \begin{tabular}{ll}
        \toprule
        & Neural Network \\
        \midrule
        Learning Rate $\alpha$ & 0.1000 \\
        Regularization $\lambda$ & 0.0001 \\
        Architecture & [22, 12, 2] \\
        Mean Accuracy & 0.9226 \\
        Mean F1-score & 0.8940 \\
        Mean Test Cost & 13.4800 \\
        \bottomrule
    \end{tabular}

        
\end{minipage}
\hfill
\begin{minipage}{0.49\textwidth}
    \centering
    \includegraphics*[width=0.9\textwidth]{./src/figures/Parkinson's_train_cost.png}
    \captionof*{figure}{\textit{Training Cost Over 19,500 Instances}}
\end{minipage}
\newpage
\begin{center}
    \texttt{Network Forrest}
\end{center}
\begin{minipage}{0.49\textwidth}
    \centering
    \includegraphics*[width=\textwidth]{./src/figures/Parkinson's Dataset_variant.png}
\end{minipage}
\hfill
\begin{minipage}{0.49\textwidth}
    \centering
    \begin{tabular}{lr}
        \toprule
         & Model 1 \\
        \midrule
        Learning Rate $\alpha$ & 0.0200 \\
        Regularization $\lambda$ & 0.0000 \\
        Mean Accuracy & 0.7542 \\
        Mean F1-score & 0.4281 \\
        Number of Iterations & 100.0000 \\
        Batch Size & 10.0000 \\
        \bottomrule
        \end{tabular}
\end{minipage}

\subsection*{Analysis}
With regard to this dataset, we found that both of our classifiers exhibited high variance, 
indicating their sensitivity to fluctuations in the training data.
In numerical datasets, such as this one, individual data points may exhibit significant variability, 
especially if they represent continuous variables with a wide range of values.
For the k-NN algorithm, variance spiked between $k$ = 1 and $k$ = 56,
representative of the overfitting issue mentioned previously.
This increase in variance suggests that the algorithm's performance deteriorated
as it tried to adapt to noise in the training data.
However, once $k$ hit 61, variance decreased substantially, 
indicating that considering a larger number of neighbours allowed the algorithm 
to smooth out the impact of individual data points and consider the dataset comprehensively.
\\

A similar spike in variance was observed in the neural network algorithm. 
Initially, as the model processed the data, 
we noticed a substantial fluctuation in the cost function across different instances, 
indicating high sensitivity to variations in the training data. 
However, the algorithm converged to a more consistent behaviour the closer it got to processing 19,500 instances,
reaching a mean test cost of 13.48.
It may be helpful to increase the complexity of the neural network architecture to capture this data variance
and learn more intricate patterns and relationships within it.
\\

The network forrest model performed poorly on this dataset, as with the titanic and loan datasets. Despite having a larger number of 
features compared to the previous two datasets, none of them stood out as particularly important to the model. The high difference between accuracy
and F1-score suggests that the model is classifying false positives at a higher rate than false negatives. This is likely due to the unbalanced dataset. The network forrest in this dataset
essentially acts as a random classifier.

\end{document}